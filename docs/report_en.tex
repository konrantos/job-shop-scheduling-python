\documentclass[12pt]{article}
\usepackage[utf8]{inputenc}
\usepackage[english]{babel}
\usepackage{graphicx}
\usepackage{amsmath}
\usepackage{float}
\usepackage{listings}
\usepackage{color}
\usepackage[a4paper, margin=2.5cm]{geometry}

\title{Job Shop Scheduling with the SPT Rule \\ \large Algorithms and Complexity Project}
\author{konrantos}
\date{}

\definecolor{codegray}{rgb}{0.5,0.5,0.5}
\definecolor{backcolor}{rgb}{0.95,0.95,0.92}

\lstdefinestyle{mystyle}{
    backgroundcolor=\color{backcolor},   
    commentstyle=\color{codegray},
    keywordstyle=\color{blue},
    numberstyle=\tiny\color{codegray},
    stringstyle=\color{red},
    basicstyle=\ttfamily\footnotesize,
    breaklines=true,                 
    captionpos=b,                    
    keepspaces=true,                 
    numbers=left,                    
    numbersep=5pt,                  
    showspaces=false,                
    showstringspaces=false,
    showtabs=false,                  
    tabsize=2
}

\lstset{style=mystyle}

\begin{document}

\maketitle

\section{Problem Description}

In the classic Job Shop Scheduling Problem (JSSP), we are given a set of jobs, each consisting of a sequence of operations that must be processed in a specific order on different machines. Each operation must be processed on a specific machine for a defined amount of time. The goal is to schedule all operations to minimize the total makespan, i.e., the time at which all jobs are completed.

\section{Analysis and Solution}

This project implements a heuristic solution using the Shortest Processing Time (SPT) rule. Each job is assigned a priority based on the sum of its operation times. Jobs with lower total processing time are scheduled earlier.

We assume non-preemptive execution and that each machine can handle only one job at a time. Machines are scheduled greedily according to the SPT job order and machine availability.

\section{Python Implementation}

The implementation reads job and machine data from standard dataset files (e.g., `la01.txt`, `mt06.txt`), which follow a specific format. For each job, both the operation durations and the machine sequence are provided.

The following Python snippet demonstrates the structure of the SPT makespan calculation:

\begin{lstlisting}[language=Python, caption=SPT-based makespan calculation]
def calculate_makespan_spt(num_jobs, num_machines, job_times, job_sequences):
    machine_schedule = [0] * num_machines
    job_completion = [0] * num_jobs
    sorted_jobs = sorted(range(num_jobs), key=lambda x: sum(job_times[x]))
    for job in sorted_jobs:
        for step in range(num_machines):
            machine = job_sequences[job][step] - 1
            time = job_times[job][step]
            machine_schedule[machine] = max(machine_schedule[machine], job_completion[job]) + time
            job_completion[job] = machine_schedule[machine]
    return max(machine_schedule)
\end{lstlisting}

\section{Results and Observations}

Experiments were conducted using standard OR-Library datasets (`la01` to `la05`, `mt06`, `mt10`, `mt20`). Below is a table showing the calculated makespan using the SPT rule versus the known optimal makespan (where available):

\begin{table}[H]
\centering
\begin{tabular}{|c|c|c|c|}
\hline
Instance & SPT Makespan & Optimal Makespan & Difference \\
\hline
la01 & 688 & 666 & +22 \\
la02 & 737 & 655 & +82 \\
la03 & 1110 & 597 & +513 \\
la04 & 938 & 590 & +348 \\
la05 & 1211 & 593 & +618 \\
mt06 & 1578 & 55 & +1523 \\
mt10 & 1616 & 930 & +686 \\
mt20 & 1852 & 1165 & +687 \\
\hline
\end{tabular}
\caption{Comparison of SPT vs. Optimal Makespan}
\end{table}

We observe that the SPT heuristic works better for smaller instances (e.g., `la01`) and becomes increasingly suboptimal for larger or more complex instances (e.g., `mt10`, `mt20`).

\section{Conclusions}

The SPT rule offers a simple and fast scheduling strategy for job shop problems. However, its performance heavily depends on the structure of the instance. While suitable for small datasets or as a baseline, it often fails to approach the optimal solution for larger or more irregular job-machine configurations.

Further optimization methods, such as tabu search, genetic algorithms, or simulated annealing, could significantly improve the makespan.

\end{document}
